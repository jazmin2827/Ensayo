\documentclass{article}
\usepackage[utf8]{inputenc}
\usepackage{graphicx}
\usepackage{color}
\begin{titlepage}
\centering
{\bfseries\LARGE Universidad de Antioquia \par}
\vspace{1cm}
{\scshape\Large Facultad de Ingeniería Electrónica \par}
\vspace{3cm}
{\scshape\Huge De la incertidumbre a lo desconocido \par}
\vspace{3cm}
{\itshape\Large Curso Informática 2 \par}
\vfill
{\Large Autor: \par}
{\Large Jazmin Andrea Moreno Castrillón \par}
\vfill
{\Large Junio 2020 \par}
\end{titlepage}
\tableofcontents
\newpage

\section{De la incertidumbre a lo desconocido}
\begin{document}

¿Existe una verdad absoluta?, ¿se está completamente seguro de la veracidad en algún aspecto independientemente del ámbito al cual pertenezca?, hace algun tiempo se pensaba que sí, específicamente a nivel matemático, y es que se tenía un modelo plenamente estructurado en el cual no había cabida para irregularidades o indeterminaciones, técnicamente todo era contabilizable, no obstante, como se ha dado a lo largo de la vida y la historia de la humanidad, se desarrollaron nuevas estructuras de pensamiento, se cuestionó los criterios que ya estaban establecidos, y como una linterna que apunta su luz hacia un sitio cuyos elementos son completamente desconocidos gracias a la penumbra del lugar, se cuestionó qué es lo que sucede con las grandes cantidades, con aquello que es tan voluptuoso que no se puede contabilizar. Para presentar una analogía un tanto ilustrativa se puede traer la paradoja del hotel infinito planteada por David Hilbert: " Dos grandes hoteleros que querían construir el hotel más grande del mundo se reunieron a dialogar sobre el asunto y comenzaron por el primer y más obvio tema a discutir: cuántas habitaciones tendría.

—¿Qué te parece si construimos un hotel con 1000 habitaciones?

—No, porque si alguien construyera uno de 2000 habitaciones, nuestro hotel ya no sería tan grande. Mejor hagámoslo de 10 000.

—Pero podría ser que alguien construyera uno de 20 000 y volveríamos a quedarnos con un hotel pequeño. Construyamos un hotel con 1 000 000 de habitaciones, ése sería un hotel grande.

—Y qué tal si alguien construyera uno con...

Como siempre podría llegar a haber un hotel más grande, llegaron a la conclusión de que era necesario hacer un hotel con habitaciones infinitas de manera que ningún otro hotel del mundo pudiera superar su tamaño."

El cuestionarse sobre este tipo de hipótesis fue una de las razones que originó la crisis de los fundamentos, y es que el hecho de tener la estricta necesidad de reestructurar el modelo matemático tal como se conocía ya imponía de por si un reto, reto asumido y potenciado por grandes pensadores.
\newpage
Godel fue una persona que realizó grandes aportes a la estructura de la matemática con su teorema de la incompletidud, si bien, desde hacía un tiempo que se venían desarrollando teoremas para lograr estructurar y entender la matemática, logró demostrar que no todos los teoremas son demostrables, puede verse un tanto paradójico, sin embargo es una de las ideas que se plantean en su fundamento.

Para ilustrar uno de los puntos principales en este teorema, el comunmente llamado "método de la autoreferencia", podemos tomar el siguiente ejemplo: “La oración posterior es cierta” y “La oración anterior es falsa”. Lo que sucede es que ciertamente se autoreferencia creando un bucle y desmontando lo que se había planteado la oración inicial.

Las ideas de Godel inicialmente no fueron bien acogidas por la comunidad científica, no obstante, no era el único ni el primero en realizarse estos planteamientos. 

Alan Turing, una persona ciertamente brillante, añadió sus aportes para indicar que existen problemas irresolubles, añadió los términos tan fundamentales y comunmente conocidos en el mundo actual como 'computación' y 'algoritmo', y es uno de los principales responsables de que la vida tal como la conocemos a nivel tecnológico y científico sea posible.
Para ilustrarnos un poco más sobre los aportes de Turing a la computación, nos ubicaremos en la 2da guerra mundial, un entorno de inteligencia militar, y es que una vez más el ambiente apocalíptico y destructivo que supone la guerra ha sido un espacio propicio para el desarrollo científico del hombre, cuyos propósitos seguramente no cumplan con los estándares morales que suponen la vida en sociedad, no obstante, históricamente han permitido que se desarrollen mecanismos de estrategia y planeación, criterios médicos y avances ingenieriles; En este caso vamos a situarnos en un evento particular, Turing era ya conocido por su intelecto, así que se le solicitó colaborar para desarrollar un mecanismo de desencriptación que permitiera tomar la información que era usada por el enemigo para comunicarse entre sí de forma secreta. Luego de un tiempo la genialidad de Turing salió a relucir con su capacidad para descifrar el funcionamiento de la "Máquina enigma", la cual tenía como función descifrar dichos mensajes mediante un algoritmo que se rige con una estructura básica, datos de entrada, proceso de datos, y salida de datos.
\includegraphics[scale=0.6]{enigma-3-rotores.jpg}
\label{fig:enigma-3-rotores.jpg}


Entre tanto Turing desarrolla un modelo bastante avanzado para su época, y es que hipotéticamente desarrolla el modelo de funcionamiento de las computadoras en general, sólo a nivel intelectual, logra desarrollar la famosa "Máquina de Turing" la cual se rige mediante la idea de tener dos cintas (infinitas), estas se encuentran divididas en casillas, lo que a su vez tiene un "cabezote" que lee la información de cada casilla y sobreescribe (dado el caso) lo que se encuentre en ella.
Si se analiza objetivamente, esta máquina es la clara estructura de nuestras computadoras, con su método de lectura, almacenamiento, procesamiento y salida.
A pesar de los enormes aportes de este personaje a la ciencia, la tecnología y la historia de la humanidad, una vez más el sesgo ideológico de las masas desestabilizó la vida de sus más grandes pensadores, Turing fue condenado por cometer actos homosexuales, y como una historia posteriormente vista en la literatura, Turing fue hallado en su laboratorio sin vida a causa de morder una manzana envenenada.
\newpage
A lo largo de la historia se dieron estudios matemáticos y avances científicos, y poco a poco la humanidad ha ido aportando hallazgos que permitieron el desarrollo de las primeras computadoras, tan enormes como una habitación. Como era de esperarse, con las primeras computadoras nacieron también los primeros problemas computacionales, más específicamente los primeros bugs.
\includegraphics[scale=0.9]{Maquina de turing.jpg}
\label{fig:Maquina de turing}

\\
La historia de los primeros bugs se remonta a 1947, cuando la ingeniera de computación y contraalmirante de la Marina de los Estados Unidos Grace Hopper notó que la computadora presentaba fallos, por lo cual verificó físicamente la máquina para encontrar una pequeña polilla atascada en uno de los mecanismos, éste hecho se documentó como un 'bug' encontrado en la computadora, y es que bug traduce literalmente 'bicho'.

Los avances computacionales permitieron que la vida tal como la conozcamos sea posible, que sea posible que seres perdidos y aparentemente solos en el universo como lo somos los humanos, tuvieran la posibilidad de explorar otros mundos, de salir de nuestro planeta, de corroborar la estructura de nuestra galaxia, de desarrollar herramientas robóticas como las sondas espaciales que permiten abrirnos paso entre lo que ya conocemos hacia lo netamente desconocido, y es que si, cuando se adquieren más conocimientos se es cada vez más consciente de lo mucho que se desconoce.
\newpage

\section{Bibliografía}

El infinito:
\textcolor{blue}{https://www.youtube.com/watch?v=SZY7ugs_DvI}

\textcolor{blue}{https://www.significados.com/simbolo-de-infinito/}

\textcolor{blue}{https://www.disfrutalasmatematicas.com/numeros/infinito.htm}

\textcolor{blue}{https://soymatematicas.com/el-infinito/}

\textcolor{black}{Crisis de los fundamentos:}

\textcolor{blue}{https://users.dcc.uchile.cl/~aabeliuk/documents/godel.pdf}

\textcolor{blue}{file:///C:/Users/Usuario/Downloads/6053-Texto%20del%20art%C3%ADculo-23401-1-10-20130521%20(1).pdf}

\textcolor{blue}{https://www.bbvaopenmind.com/ciencia/matematicas/asi-termino-el-sueno-de-las-matematicas-infalibles/}

\textcolor{black}{Programa de Hilbert:}

\textcolor{blue}{http://rebeccagoldstein.com/books/incompleteness/index.html}

\textcolor{blue}{http://plato.stanford.edu/entries/hilbert-program/}

\textcolor{black}{Máquina enigma}

\textcolor{blue}{https://hipertextual.com/2011/07/la-maquina-enigma-el-sistema-de-cifrado-que-puso-en-jaque-a-europa}

\textcolor{black}{Máquina universal:}

\textcolor{blue}{https://www.youtube.com/watch?v=iaXLDz_UeYY}

\textcolor{black}{Georg Cantor:}

\textcolor{blue}{https://www.bbc.com/mundo/noticias-45300219.}

\textcolor{black}{Godel:}

\textcolor{blue}{https://www.bbc.com/mundo/noticias-43568588}

\textcolor{blue}{https://elpais.com/elpais/2019/01/24/ciencia/1548329597_971134.html}

\textcolor{black}{Alan Turing: }

\textcolor{blue}{https://www.lavanguardia.com/historiayvida/historia-contemporanea/20180611/47312986353/que-aporto-a-la-ciencia-alan-turing.html}

\end{document}
